%% * Header

\documentclass{article}

%% ** Packages

\usepackage{natbib}
\usepackage[utf8]{inputenc}
\usepackage[francais]{babel}
\usepackage{amsthm, amssymb, amsmath}
\usepackage{hyperref}

%% ** Shorthands

\newcommand{\todo}[1]{\textcolor{red}{#1}}
\newcommand{\ie}{\textit{i.e.}\ }
\newcommand{\vs}{\textit{vs.}\ }
\newcommand{\eg}{\textit{e.g.}\ }

\newcommand{\TODO}[1]{\textbf{TODO:} #1}
%\newcommand{\TODO}[1]{}

%% ** Theorem styles

\newtheorem{theorem}{Théorème}
\newtheorem{proposition}{Proposition}
\newtheorem{lemma}{Lemme}

\theoremstyle{definition}
\newtheorem{definition}{Définition}
\newtheorem{example}{Exemple}

\theoremstyle{remark}
\newtheorem{remark}{Remarque}
\newtheorem{para}{} 

%% ** Title

\title{Implémentation de la \\
       théorie des types dépendants}
\author{Roman Delgado}
\date{}

\begin{document}

\maketitle

%% ** Abstract

\begin{abstract}
Ce document représente la synthèse de mes premières lectures ainsi que
de ma compréhension du sujet.  Ce projet consiste en l'implémentation
d'un système de type.  Il sera donc question de définir les principaux
concepts que j'ai pu aborder lors de mes lectures.  La seconde partie
se concentre sur les questions liées à l'implémentation du lambda
calcul et fait office de travail préliminaire au projet.
\end{abstract}

\tableofcontents

\clearpage

%% * Main

\section{Théorie}


\subsection{Lambda calcul non typé}

Alonzo Church introduit le lambda calcul non
typé~\citep{church:lambda-calcul} en 1936. Le lambda calcul est un
modèle de calcul universel au même titre que les machines de Turing.
A l'aide d'une syntaxe simple, il permet d'étudier les propriétés du
calcul de façon formelle.  Les fonctions ne portent pas de nom et ne
sont applicables qu'à un seul argument. Ce système est très souple car
tout est fonction, on se concentre donc uniquement sur la partie
calculatoire.  Le lambda calcul non typé a connu un formidable succès,
donnant lieu à de multiples variantes dont on trouvera une
présentation moderne et synthétique dans l'ouvrage
de~\citet{pierce:tapl}.

\subsubsection{Syntaxe}


On peut définir le lambda calcul non typé de la façon suivante :
%
\begin{align*}
t &::= & \mbox{(lambda terme)} \\
  &|\quad x & \mbox{(variable)} \\
  &|\quad \lambda x. t & \mbox{(abstraction)} \\
  &|\quad t\: t          & \mbox{(application)}
\end{align*}

Pour faire une analogie avec les languages de programmation, on peut
dire qu'une abstraction est une définition de fonction et que
l'application consiste à appliquer une fonction à un argument (qui
peut également être une fonction, on parle alors de fonction d'ordre
supérieur).

\begin{example}
 On peut donc définir un lambda terme représentant la fonction identité
 de la façon suivante :
%
\[ 
  \lambda x. x
\]
\end{example}

\subsubsection{Sémantique}

Avant de parler de la réduction, il faut pouvoir déterminer si une
variable est liée ou libre. Une variable est \emph{liée} lorsque celle
ci apparait à l'intérieur d'une abstraction. Inversement, une variable
non déclarée par une abstraction sera dite \emph{libre}.

\begin{example}

 Dans le lambda terme \(\lambda x. x\: y\), la variable \(x\) est liée
 tandis que la variable \(y\) est libre.
\end{example}

\begin{example}
  Dans le lambda terme \((\lambda x. \lambda y. x\: y)\: y\), la
  variable \(x\) est liée ainsi que la première occurence de la
  variable \(y\). Cependant, la variable \(y\) à l'exterieur des
  parenthèse est libre.
\end{example}

Un terme toutes les variables sont liées est dit \emph{clos}.

La règle de \(\beta\)-réduction consiste à appliquer la transformation
suivante dans un terme
%
\[
(\lambda x. t)\: u \leadsto t[x \mapsto u]
\]

On ne peut donc effectuer une réduction que sur les applications dont
le membre de gauche est une abstraction : on nomme \emph{redex} les
sous-termes de cette forme. Réduire un terme consiste à substituer
dans le membre de gauche la variable liée par le terme de droite.  Le
``calcul'' d'un terme correspond alors à une série de réductions de ce
terme.


\begin{example}

  La fonction identité appliquée à un terme \(t\) réduit vers ce même
  terme \(t\) :
  %
  \[
  (\lambda x. x) t \leadsto t
  \]
\end{example}

Un lambda terme en \emph{forme normale} est un terme pour lequel
aucune réduction n'est possible. Le lambda calcul étant un modèle de
calcul universel, il permet de décrire des termes/programmes qui ne
terminent pas : ceux-ci n'ont pas de forme normale.

\begin{example}
 
  Le terme suivant n'admet pas de forme normale :
  %
  \begin{align*}
  (\lambda x. x\: x) (\lambda x. x\: x) &\leadsto (x x)[x \mapsto \lambda x. x\: x] \\
                                        &\leadsto (\lambda x. x\: x) (\lambda x. x\: x) \\
                                        &\leadsto \ldots \\
  \end{align*}
\end{example}

\subsubsection{Les entiers de Church}

Malgré son apparente simplicité, le lambda calcul permet de définir
énormément de concepts intéressants. Ici, nous allons comment créer
les entiers naturels \textit{ex nihilo} en utilisant un \emph{codage à
  la Church}.

L'idée consiste à représenter le nombre \(n\) par une fonction d'ordre
supérieur prenant en argument une fonction \(f\) et l'appliquant \(n\)
fois. On définira donc 
%
\begin{align*}
\mathsf{zero} &= \lambda f. \lambda x. x \\
\mathsf{un}   &= \lambda f. \lambda x. f\: x \\
\mathsf{deux} &= \lambda f. \lambda x. f\: (f\: x)
\end{align*}

Afin de construire tous les entiers naturels, on s'inspire alors de la
définition des entiers de Peano et définit le successeur d'un nombre \(n\)
comme
%
\[
\mathsf{successeur} = \lambda n. \lambda f. \lambda x. n\: f\: (f\: x)
\]
%
c'est-à-dire \(n\) applications de \(f\) précédées d'une première
application de \(f\), soit \(n+1\) applications.

De la même manière, on construit l'addition de deux nombres \(m\) et
\(n\) en faisant \(m\) applications répétées de \(f\) précédée par
\(n\) applications :
%
\[
\mathsf{plus} = \lambda m. \lambda n. \lambda f. \lambda x. m f (n f x)
\]


\subsection{Lambda calcul simplement typé}

\TODO{Ici il faut faire une transition avec le fait que le lambda
  calcul non typé correspond au typage dynamique, alors que le lambda
  calcul est la version typage statique.}


Le lambda calcul simplement typé~\citep{church:simple-type} est une
extension du lambda calcul. Celui ci perd en dynamisme et permet
d'effectuer moins de calcul mais possède l'avantage de soulever les
problèmes liés à la classification des données.


\subsubsection{Règles de typage}

\TODO{Le lambda calcul non typé permet de formaliser une infinité de ....
 mais l'absence de type ne lui permet pas de vérifier la validité de ses
 calculs.}

Pour tout programmeur il parait logique que l'expression
%
\[
 \mathsf{if}\: 2\: \mathsf{then}\: \mathsf{true}\: \mathsf{else}\: \mathsf{false}
\]
%
n'a pas de sens car il faudrait que le résultat de l'expression de la
condition soit un booléen et non un entier. Les types permettent de
palier à ce type d'erreur en les détectant à la compilation.


Les \emph{types} sont décrits par la grammaire suivante :
%
\begin{align*}
  T &::=             & \mbox{(type)} \\
    &|\quad \mathsf{bool} & \mbox{(booléens)} \\
    &|\quad T \rightarrow T   &  \mbox{(fonction)} 
\end{align*}

La vérification de type s'effectue dans un \emph{contexte} qui assigne
à chaque variable d'un programme son type. Le contexte est donc une
liste ordonnée de paires variable/type :
%
\begin{align*}
  \Gamma &::= &\mbox{(contexte)} \\
         &\quad . & \mbox{(contexte vide)} \\
         &\quad \Gamma, x : T & \mbox{(type de variable)} 
\end{align*}

Les règles de typage sont définies en Figure~\ref{fig:typage-simple}.

\begin{figure}

\begin{align*}
&  \cfrac{x:T\,\in\Gamma}
         {\Gamma\vdash x:T}\mbox{(Var)} 
  \\
&  \cfrac{\Gamma, x:A \vdash t:B}
         {\Gamma \vdash \lambda x. t \,:\, A\rightarrow B}\mbox{(Abs)} \\
&  \cfrac{\Gamma \vdash f : A\rightarrow B \quad
          \Gamma \vdash s : A}
         {\Gamma\vdash f\: s : B}\mbox{(App)}
\end{align*}

\caption{Typage (lambda calcul simplement typé)}
\label{fig:typage-simple}
\end{figure}

La règle (Var) spécifie le typage des variables. Cette expression se
lit en partant du numérateur pour déduire le dénumérateur : supposons
que la variable x de type T est présente dans le contexte, alors on
conclut que le type de la variable x est T.

La règle (Abs) spécifie le typage des lambda
abstractions. L'abstraction construit un type de la forme \(A
\Rightarrow B\) : il faut donc que la variable associée au lambda
terme soit de type \(A\) et le résultat de type \(B\).

\TODO{Justifier la règle App}
               

\subsubsection{Sémantique statique}

Le typage permet d'obtenir, à la compilation, des garanties quant à la
bonne exécution des programmes.  Formellement, ces garanties se traduisent en deux propriétés essentielles d'un système de type :
\begin{description}
\item[``Progress'':] soit \(t\) un terme correctement typé alors \(t\) est une valeur ou il
 existe une réduction telle que \(t \leadsto t'\)
\item[``Preservation'':] si un terme est correctement typé alors il existe une règle
 de dérivation pour ce terme, le terme obtenu est lui aussi correctement
 typé.
\end{description}

La conjonction de ces deux propriétés donne le slogan ``un programme
bien typé ne plante jamais''\citep{milner:no-wrong} : tout programme
bien typé réduit vers une valeur générant, à chaque étape de
réduction, un terme bien typé.


\subsection{Types dépendants}

Afin d'introduire les types dépendants prenons l'exemple suivant: soit
la fonction ayant pour but d'effectuer un ``et logique'' entre deux
vecteurs de bits de même taille.

Dans un système de type statique on ne peut pas faire de restriction
quant à la taille que feront les deux vecteurs.  
\TODO{présenter le code en OCaml}

\begin{verbatim}
 code code code
\end{verbatim}

Grâce à la théorie des types dépendants il est possible de demander
explicitement deux vecteurs de taille \(n\) et donc d'effectuer le
``et logique'' sans problème. 
\TODO{présenter le code Agda}

\begin{verbatim}
 code code code
\end{verbatim}


Les types dépendants permettent donc de caractériser un objet non
seulement par son type mais aussi par sa valeur et ainsi de créer des
règles de typage plus complexes et offrant plus de souplesse que le
typage simple.

\subsubsection{Règles de typage}

\TODO{Ajouter une référence vers \citep{hottbook,thompson:types-fp}}

Dans cette partie, il sera question de présenter un système de type
dépendant comprenant quelques types de base simple ainsi que leurs
règles d'introduction et d'élimination.

Nous considérons les types de base suivant: le type booléen
(\textsf{bool}), l'ensemble vide (\textsf{void}) ainsi que l'ensemble
unitaire (\textsf{unit}). \TODO{Les introduire dans le lambda calcul
  simple plutôt et réutiliser tel quel ici.} Ces règles étant
similaires, nous concentrons notre attention sur le type booléen. Les
règles d'introduction (BoolI1) et (BoolI2) sont assez simples puisque
peu importe le contexte \textsf{true} et \textsf{false} sont des
booléens.  La règle d'élimination des booléens correspond au
branchement condition \textsf{if} \ldots \textsf{then} \ldots
\textsf{else} \ldots puisque cette règle permet d'utiliser nos
booléens précédement définis.

\begin{figure}
  \[\begin{array}{ccc}
  \cfrac{}
        {\Gamma\vdash \mathsf{bool} \in \mathsf{Type}}\mbox{(BoolF)}
  & 
  \begin{array}{c}
  \cfrac{}
        {\Gamma \vdash \mathsf{true} \in \mathsf{bool}}\mbox{(BoolI1)}
  \\
  \cfrac{}
        {\Gamma \vdash \mathsf{false} \in \mathsf{bool}}\mbox{(BoolI2)}
  \end{array}
  &
  \cfrac{\Gamma \vdash t \in \mathsf{bool} \quad 
         \begin{array}{l}
          \Gamma \vdash u \in A \\
          \Gamma \vdash v \in A 
         \end{array}}
        {\Gamma \vdash \mathsf{if}\: t\: \mathsf{then}\: u\: \mathsf{else} v \in A}
  \\
  \cfrac{}
        {\Gamma\vdash \mathsf{unit} \in \mathsf{Type}}\mbox{(UnitF)}
  & 
  \cfrac{}
        {\Gamma\vdash () \in \mathsf{unit}}\mbox{(UnitI)}
  \\
  \cfrac{}
        {\Gamma\vdash \mathsf{void} \in \mathsf{Type}}\mbox{(VoidF)}
  &
  &
  \cfrac{\Gamma\vdash t \in \mathsf{void}}
        {\Gamma\vdash t \in A}\mbox{(VoidE)}
  \\
  \cfrac{\begin{array}{c}
          \Gamma\vdash A \in \mathsf{Type} \\
          \Gamma, x : A \vdash B \in \mathsf{Type}
         \end{array}}
        {\Gamma \vdash (x:A) \rightarrow B \in \mathsf{Type}} 
        \mbox{(ImplF)}
  &
  \cfrac{\Gamma, x:A \vdash b\in B}
        {\Gamma \vdash \lambda x. b \in (x:A) \Rightarrow B}
        \mbox{(ImplI)}
  & 
  \cfrac{\begin{array}{c}
          \Gamma f \in (x:A)\Rightarrow B \\
          \Gamma \vdash s \in A
         \end{array}}
        {f\: s \in B[x \mapsto s]}
        \mbox{(ImplE)}
  \end{array}\]
  \caption{Typage (lambda calcul dépendant)}
  \label{fig:typage-dependant}
\end{figure}


\section{Implémentation}

Dans cette partie il est question de l'implémentation du lambda calcul
non typé en précisant les choix de réalisation.  Le language choisi
pour cette implémentation et qui le restera pour l'ensemble de ce
projet est Ocaml, un language fonctionnel dont la syntaxe permet de se
rapprocher au plus des définitions mathématiques, ce qui rend le
passage des définitions théoriques aux implémentations plus aisées.


\subsection{Lieurs}

Le premier problème auquel j'ai été confronté fut la représentation
qu'il fallait donner des variables libres ainsi que des variables
liées.  D'un point de vue algorithmique lors du parcours d'un terme
determiné si une variable est liée peut s'avérer complexe et
nécéssiterait de nombreuses opérations supplémentaires.  Une solution
classique~\citep{pierce:tapl} consiste à ne pas définir les variables
par des charactères mais avec des entiers. C'est ce que l'on appel les
\emph{indices de de Bruijn}.

Considérons le terme \(\lambda x. \lambda y. x\: y\) peut se
représenter comme \(\lambda. \lambda. 1\: 0\).  Dans cette
représentation les variables sont représentées en fonction du nombre
de lieurs les précédant.  Ici, la variable \(y\) est représentée par
l'entier 0 car elle est liée au premier lieur à sa gauche. Même
raisonnement pour la première occurence de \(x\).

Cependant, on souhaiterait également pouvoir représenter des termes
ayant des variables libres. Nous adoptons pour cela la convention
\emph{localement
  anonyme}~\citep{mcbride:not-a-variable,chargueraud:locally-nameless}
(``locally nameless''). Considérons le terme \((\lambda x. x) y\) : la
variable \(y\) est une variable libre que l'on représentera de façon
nommée, sous la forme de caractère. Ce terme s'écrit donc
\((\lambda. 0)\: y\).

Voici donc l'implémentation que nous avons choisi des lambda terme en Ocaml :
%
\begin{verbatim}
type lambda_term =   
   | FreeVar of string    
   | BoundVar of int    
   | Abs of lambda_term   
   | Appl of (lambda_term * lambda_term)
\end{verbatim}

Afin de s'implifier la lecture des lambda terme voici une fonction permettant
 de convertir un lambda terme en chaîne de caractère.
%
\begin{verbatim}
let rec lambda_term_to_string t = 
   match t with 
   | FreeVar v -> v 
   | BoundVar v -> string_of_int v 
   | Abs x -> "[]." ^ lambda_term_to_string x 
   | Appl (x,y) -> "(" ^ lambda_term_to_string x ^ " " ^ lambda_term_to_string
 y ^ ")"
\end{verbatim}

\subsection{Substitution et réduction}

Il faut maintenant définir l'opération de réduction. Comme expliqué
précédemment, cela équivaut à substituer les occurrences des variables
liées dans l'abstraction de gauche par les termes de droite.  J'ai
donc choisi de définir tout d'abord une fonction récursive de
substitution afin de pouvoir l'utiliser plus tard dans la fonction de
réduction.  
% 
\begin{verbatim}
let rec substitution t indice tsub = 
   match t with 
   | FreeVar v -> FreeVar v 
   | BoundVar v -> if v = indice then tsub else BoundVar v 
   | Abs x -> Abs(substitution x (indice+1) tsub) 
   | Appl (x,y) -> Appl(substitution x indice tsub,substitution y indice tsub)
\end{verbatim}

La subtilité de cette fonction réside dans le fait que à
chaque fois que l'on rencontre un lieur il faut appeler la fonction
substitution mais cette fois ci avec l'indice de la variable liée
incrémenté de un.

On ne peut appliquer une réduction qu'à un terme dont le membre de
gauche est une abstraction.  Cette fonction de réduction se contente
simplement d'appeler la fonction substitution seulement sur les redex.
 
\begin{verbatim}
let reduction t = 
   match t with 
   | FreeVar v -> FreeVar v 
   | BoundVar v -> BoundVar v 
   | Abs x -> Abs(x) 
   | Appl(Abs(x),y) -> substitution x 0 y 
   | Appl(x,y) -> failwith "erreur"
\end{verbatim}

Ici les applications n'étant pas considérées comme réductibles on
renvoie une erreur.  Ce choix d'implémentation sera expliqué par la
suite lorsque nous aurons défini la fonction d'évaluation.  Afin de
réduire un terme jusqu'à sa forme normale si elle existe peut être
considérée comme une opération de normalisation, mais étant donné que
certains termes ne peuvent pas se réduire nous avons décidé d'appeler
cette fonction évaluation. 

Il existe de nombreuses stratégies pour calculer les lambda termes,
nous avons choisis la stratégie de reduction
normale~\citep{pierce:tapl}, aussi appelée ``appel par nom''.  Cette
stratégie consite à réduire le redex le plus à gauche. Voici donc une
implémentation de cette stratégie :

\begin{verbatim}
let rec evaluation t =
   match t with  
   | FreeVar v -> FreeVar v  
   | BoundVar v -> BoundVar v  
   | Abs x -> Abs(x) 
   | Appl(Abs(x),y) -> evaluation(reduction t) 
   | Appl(BoundVar x,y) -> Appl(BoundVar x,y)  
   | Appl(FreeVar x,y) -> Appl(FreeVar x,y)  
   | Appl(x,y) -> evaluation(Appl(evaluation x, y))
\end{verbatim}

Avec l'ensemble de ces fonctions nous avons donc implémenté le lambda
calcul non typé.


\subsection{Types dépendants}


La suite du projet consistera donc comme le sujet l'indique en une
implémentation d'un système de type dépendant en Ocaml.

\TODO{Expliquer le challenge : système de type = relation, type checker = fonction}
\TODO{Citer \citep{loeh:tuto-dependant,coquand:mini-tt}}


%% * Outro

\bibliographystyle{abbrvnat}
\bibliography{prerapport.bib} 


\end{document}
